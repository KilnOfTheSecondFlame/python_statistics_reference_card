\documentclass[14pt]{article}

\usepackage[a4paper,margin=1cm,landscape]{geometry}
\usepackage[utf8]{inputenc}
\usepackage{amsmath}
\usepackage{amsfonts}
\usepackage{amssymb}
\usepackage{graphicx}
\usepackage{multicol}
\usepackage{lipsum}
\usepackage{minted}

\setlength{\parindent}{0cm}

\begin{document}
\begin{center}
	\section*{Python-Referenzkarte}
\end{center}
\begin{multicols}{3}

\subsection*{Bibliotheken}
\begin{minted}{python}
import matplotlib.pyplot as plt
import pandas as pd
import numpy as np
\end{minted}

\subsection*{Pandas Series}
\begin{minted}{python}
series = pd.Series([79.98, 80.04, 80.02])
series = pd.Series(
	[1, 5, 9, 15, 20],
	index=("mo", "di", "mi", "do", "fr")
)
\end{minted}

\mintinline{python}{series.sum()}: Die Summe der Elemente von $series$ \\
\mintinline{python}{series.mean()}: Der Durchschnitt der Elemente von $series$ \\
\mintinline{python}{series.median()}: Der Median der Elemente von $series$ \\
\mintinline{python}{series.var()}: Die Varianz der Elemente von $series$ \\
\mintinline{python}{series.std()}: Standardabweichung von $series$ 

\begin{minted}{python}
series["mi"]
series.index
series.size
\end{minted}

\subsection*{Quantile und Quartilsdifferenz}
\mintinline{python}{series.quantile(q=0.25, interpolation="midpoint")}
Quantile (z.B. 25\%, 75\%, \dots) von $series$ \\

\mintinline{python}{q75, q25 = series.quantile(q = [.75, .25], \ } \\
\mintinline{python}{	interpolation="midpoint")} \\
\mintinline{python}{iqr = q75 - q25} \\
Quartilsdifferenz

\subsection*{Numpy Basics}
\begin{minted}{python}
np.linspace(start=1, stop=2, num=4)
np.arange(start=1, stop=7, step=.6)
np.sqrt(2)
np.square(2)
np.round(2.35, decimals=1)
\end{minted}

\subsection*{Numpy Arrays}
\begin{minted}{python}
x = np.array([2, 1, 4, 5, -8])
3*x
x*x
\end{minted}

\subsection*{Pandas DataFrame}
\begin{minted}{python}
temp = pd.DataFrame({
	"Luzern": ([1,5,9,15,20,25,25]),
	"Basel": ([3,4,12,16,18,23,32]),
	"Zuerich": ([8,6,10,17,23,22,24])},
	index=["jan","feb","mar","apr","mai","jun","jul"]
)
temp.columns
temp.mean(axis=1)
temp.loc["mai":"jul","Luzern"]
temp.loc[["mai","jul"],["Basel","Zuerich"]]

mean = data.mean()['Physical.activity']
data.loc[data['Physical.activity'] < mean, :]['Physical.activity']

data = pd.read_csv(r"*child.csv", sep=",", index_col=0)
data.shape
data.describe()
data.nsmallest(1, 'Infant.mortality')
data.sort_values(by='Drunkenness', ascending=False).index[0:5]
data.head()
\end{minted}

\subsection*{Matplotlib PyPlot}
\begin{minted}{python}
plt.subplot(221)
series.plot(kind="hist", edgecolor="black")
plt.title("Histogramm von Methode A")
plt.xlabel("Methode A")
plt.ylabel("Haeufigkeit")
plt.show()
series.plot(kind="hist", normed=True, edgecolor="black")
geysir["Zeitspanne"].plot(kind="hist", bins=20,edgecolor="black")
geysir["Eruptionsdauer"].plot(kind="hist", normed=True, cumulative=True, edgecolor="black")
\end{minted}

\end{multicols}
\end{document}

